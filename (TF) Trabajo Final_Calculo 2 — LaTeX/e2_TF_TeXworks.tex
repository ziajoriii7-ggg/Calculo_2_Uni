\documentclass{article}
\usepackage[left=0.7in, right=0.7in, top=0.7in, bottom=0.7in]{geometry}
\usepackage{relsize}
\begin{document}
\larger[2] 

\textbf{Ejercicio 2}\par

Una partícula parte de la posición $ r(0) =  \langle a; b; c\rangle$. Si se sabe que su velocidad en cualquier instante t es $ v(t) = 2 cos t i (3t^2 + 1)j + (4t^3 + t)k$. \par
a. Determine la posición de la partícula para cualquier instante de tiempo. \par
b. Determine la aceleración de la partícula para cualquier instante de tiempo. \par
\end{document}
 